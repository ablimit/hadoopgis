\section{Runtime Scheduling}

In addition to building the runtime system infra-structure described in last
section, it will be necessary to provide a hierarchical scheduling to assign
task from the \emph{Manager} to the \emph{Workers}, called Manager Level
Scheduling, as well as internally to each \emph{Worker} --- Worker Level Scheduling. 

\subsection{Manager Level Scheduling}

The main objective is to assign tasks to \emph{Workers} in order to optimize
the performance of the overall system. There are two major aspects that should
be considered:~(i)~Load imbalance among \emph{Workers}, and a demand-driven
task assignment among Workers and Manager should be enough for solving this
problem; (ii)~A smarter choice of which task to assign for a Worker could also
potentially improve the performance. For instance, if a Worker already has a
number of tasks with low speedup, it would be better to sent a task with high
speedup for that Worker, which may be able to make better local scheduling
decisions. This tends to be better than traditional FCFS scheduling, as we
observed in the PRIORITY compared to FCFS.  Below is the list of features that
should be added to the Manager scheduler.

\begin{itemize}
	\item Demand-driven tasks assignment;

	\item Multi-task assignment to overlap communication/computation;

	\item Performance Aware task choice;
\end{itemize}

\subsection{Worker Level Scheduling}

The Worker level scheduler is more aligned with what we already proposed for 
the IPDPS paper, with addition of dependency among tasks, and potentially new
scheduling techniques that take into account the number of dependencies when
choosing a task to executed. The summary of features is:

\begin{itemize}
	\item Support to dependency among sub-tasks;

	\item Scheduling policies similar to those described in~\cite{Teodoro-IPDPS2012};

	\item May include number of tasks dependencies into the scheduling decision;

\end{itemize}


\subsection{Analytical Performance Model}

\begin{itemize}

	\item Set of tools to analytically estimate applications' performance.
We already have a draft that we intended to use with the HPDC, before it turn
into Morphological Reconstruction;

\end{itemize}

