\section{Learning to Schedule}

In our work, we have so far asked the programmer to provides us with
information about the expected performance (speedup) of tasks when executed in
the GPU.  The idea here is to start working towards an automation of the
process learning the performance of application (stages of the application),
and use this information to automatically schedule the application in CPU-GPU
environments.

A few works have started addressing this problem on the context of modeling and
estimation of GPU application's performance, most of them using history based
strategy based on application
profiling~\cite{Augonnet:2009:ACP:1884795.1884805,gregg10contention,5645393,Jimenez:2008:PRC:1505816.1505822,Kicherer:2011:CFM:1944862.1944883}.
Most of these works, however, fail miserably to provide a generic approach, and
are very simplistic though they already show that doing such a profiling based
performance estimation is promising. A related interesting work for
multicores CPUs only is~\cite{LUO:2009:INRIA-00436034:1}, where the authors
take advantage of profile history to estimate the set of optimizations that
would result into the best performance for an application. In my understanding,
we could leverage some of their propositions in our work. Moreover, there is
also some work on performance modeling~\cite{1138092,1498789} which states that
computing relative performance among devices (different processors or disks,
etc) is easier than trying to estimate execution times. This is good news once
our scheduling policies are based on speedups. Another good news is that the
accuracy of the speedup estimates is not critical as long as the order of tasks
is not affected, as we stated for IPDPS. Also, a estimation that separates
tasks with small and large speedups should be enough to achieve good
performance.


To accomplish this task we will need the following components.

\begin{itemize}
	\item Knowledge Database: to store performance data collected;

	\item Performance fitting algorithms: to estimate performance from the history data stored at the Knowledge database;

	\item Automated task grouping: is one of the features that could be derived from this knowledge.
\end{itemize}



